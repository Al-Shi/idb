\documentclass[10pt]{article}

\usepackage{amsmath, amsfonts, amssymb}

\title{Implementation of Databases Sheet 1}
\author{Ilya Kulikov 351063, Alina Shigabutdinova, Oleg Chernikov}

\begin{document}
  \maketitle
  \section*{Exercise 1.1}
  1. There are layers in the database system:
  \begin{itemize}
    \item Logical data structures: this layer contains auxiliary structures:
    external schema description and integrity rules. The goal of this layer is
    to translate and optimize queries. Interface with transaction programs is
    SQL, which operates with relations, views and tuples.
    \item Logical access structures: this layer contains auxiliary structures:
    access path data, internal schema description. The task of this layer is to
    manage cursor, sort components and dictionary. Interface with logical data
    structures is record oriented interface, which operates with records, sets,
    keys, access paths.
    \item Storage structure: this layer contains auxiliary structures: DBTT, FPA,
    page indexes etc. The task of this layer is to manage record and index in the
    database system. Interface with Logical access structures is Internal record
    interface, which operates with records, B* trees etc.
    \item Page assignment: this layer contains auxiliary structures: page and
    block tables. The task of this layer is to manage buffer and segments. The
    interface with Storage structure is System buffer interface, which operates
    with pages and segments.
    \item Memory assignment structures: this layer contains auxiliary structures:
    VTOC, extent tables, system catalogue. The task of this layer is to manage files
    and external memory. The interface with Page assignment is File interface,
    which operates with blocks and files. The interface with Physical volume is
    Device interface, which operates with tracks, cylinders, channels etc.
  \end{itemize}
  2. The following sequence matches the top-down architecture:\\
  (E) $\Rightarrow$ (B) $\Rightarrow$ (D) $\Rightarrow$ (A) $\Rightarrow$ (C)\\\\
  3. (a) Data independence means that application or some external representation of
  the data is independent from internal storage.\\
  It is not definitely given in the slides, but Data Independence contains
  two parts: Logical Data Independence and Physical Data Independence. Logical Data
  is a metadata and data about database itself, i.e. it describes how the actual
  data stored and managed in the database. So Logical Data Indeendence is a kind of
  mechanism, which acts independently from actual data stored on the disk. For example,
  if we do some change on table format, it should not change the data on the disk.\\
  Physical data independence is the mechanism which allows to change physical data
  without impact the schema or some other logical data.\\
  (b) Data independence is an important feature because it is not feasible to store
  the same data for each application several times and to provide shared data to
  several applications and to be sure, that data will be accessible and consistent.
  More over, with data independence it is possible to use logical data structures
  and descriptive queries.\\
  (c) Data structures and smart modern interfaces hiding all low-level layers and working
  with filesystems and block devices.

\end{document}
